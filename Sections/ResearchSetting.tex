\section{Research Setting}
\label{sec:Research Setting}
This section introduces four research questions along with the motivation for this question. In this section, we will describe the research questions.

\subsection{Research Questions}
\noindent \textbf{RQ1.} What are the difficult topics in the questions of new languages in Stack Overflow?

\indent  Developers of new languages face problems that are rarely answered or get \emph{delayed answers}. By the \emph{delayed answer}, we imply that answer which is accepted by the user and received after the median answer interval of that month. We want to know about these questions so that special measures can be taken to answer this question.

\noindent  \textbf{RQ2.} When can we expect the availability of adequate resource of the new languages in Stack Overflow?

\indent After the introduction of a new language, the resources of those languages may be absent in QA sites. Gradually the lackings will be met. We want to know the time interval after which we can expect the availability of these resources of new languages in Stack Overflow at a satisfactory level.

\noindent  \textbf{RQ3.} Is there any relation between the growth of a language and developers' activity pattern of that language?

\indent Stack Overflow has become one of the most prominent QA sites over the years. It has been used as a source to gain insight into developers' activity\citep{Ahmed2017}. We can observe developers' activity from the frequency of question and answer in Stack Overflow and also from the number of developers and repositories of that language in GitHub. GitHub provides \emph{issue}\citep{GithubIssue} to keep track of task, bug and feature request for a project. Most of the issues of a GitHub project are associated with bugs or feature request\citep{Bissyande2013}. Therefore, it can be assumed that the solution to the issues leads to the advancement of the project. Hence, issues reflect the growth of the project. Moreover, every new release of a language implies the growth of that language. Thus, developers activity pattern after a new release of that language can also help to understand the relationship. We want to understand the relation between the growth of a programming language and developers' activity pattern.

\noindent  \textbf{RQ4.} What are the characteristics of answer pattern for new languages in Stack Overflow?

\indent With the evolution of a new language, the number of skilled developers increases. Being the most used programming related QA site, Stack Overflow is supposed to reflect that change. The increasing number of skilled developers may have its impact on answer patterns such as the expected interval of first and accepted answers, unanswered question ratio. By answer interval, we imply the delay between a question posted and an answer received. We want to know the characteristics of the answer pattern of new languages.

\noindent  \textbf{RQ5.} Are the questions of new languages in Stack Overflow answered mostly by the developers of predecessor languages?

\indent Stack Overflow includes developers from a diverse domain. We are interested to see if there is a pattern that the experts in the predecessor of a new language mostly answers the questions of that language (e.g., Objective-C for Swift).

\iffalse


\noindent  \textbf{RQ6.} Is there any change over time in topics of Stack Overflow questions with the growth of the relevant language?

\indent Evolution of a language has many phases. These phases must have their footprint in the topics of the questions asked by the developers of the new languages. Hence, we want to know the topics of the questions asked by developers with respect to a timeline so that we can identify those phases.

\noindent \textbf{RQ7:} \emph{How do developers respond to the release of a new version of a language?}

\indent A new release of a language comes with a new and updated feature set. These updated or new features must have some impact over developer community. It may trigger new thread of question in Stack Overflow. We want to know the after effect of such release.
\fi
