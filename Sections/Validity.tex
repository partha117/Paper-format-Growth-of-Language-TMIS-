\section{Threats to validity}
\label{sec:validity}
In this section, we discuss the validity of our study.

\indent \textbf{Internal validity:} Use of tags to categorize questions by language is an internal threat to validity. A new Stack Overflow user may not add an appropriate tag with the question. However, Stack Overflow questions go through an extensive moderation process and eventually, it will have the appropriate tags. In some cases, our identification of posts by tags may not be able capture the posts of new languages. To alleviate this threat, we considered the relevance of tags.  In this study, we have used Stack Overflow as the primary dataset. There are many language specific developers' forum and QA sites and it is possible that those sites may contain posts that can help to understand the growth of new languages. However, we believe that a large number of participants and the widespread popularity of Stack Overflow has made it a familiar venue for developers. Hence, the posts of Stack Overflow are considered enough to understand the trends of growth of new language.

\indent \textbf{External validity:} After the inception of the Stack Overflow (2008), about 35 programming languages have been released\citep{wiki:Timeline} whereas this study is focused on three languages (Swift, Go, and Rust). For this reason, our research results may not apply to other new languages. However, in this study, we did not emphasize any specific feature of a particular language. The languages we considered vary in terms of their time of inception and other properties (such as having predecessor language or not). Instead, we focused on the characteristics and trends of the growth of new languages. We compared the growth trends with top-tier (Java) and mid-category (Python) languages and found that mid-category language (Python) shows similar characteristics. The reason for the dissimilarity with high-level languages is that we missed the community interaction at the initial period of this language. Java was published a long time ago and was already a developed language before the establishment of SO. Therefore, we think our findings are free from any bias on particular language.

\iffalse

\indent \textbf{Construct validity:} In this study, votes of accepted answers are given double weight in the calculation of post quality. The weight used may not represent their exact contribution. However, the magnitude of weight do not influence our analysis. Thus, double weight in the accepted answer will not invalidate our claim.
\fi