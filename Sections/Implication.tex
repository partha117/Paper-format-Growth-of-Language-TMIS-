\section{Implication}
\label{sec:Implication}
Thus far, we have discussed the characteristics of answer pattern of new languages, the relation between the advancement of new languages and its developers' activity, expected answer interval for new languages, and predecessor language of interest of the respondents of new languages. In this section, we discuss the implications of our findings. As well as helping developers to find resources while learning a new language, our study can also help language owners, researchers, and Stack Overflow to refine their strategies to support the growth of new languages. 

\indent \textbf{Developers:} In this study, we have estimated the answer interval and time when we can expect the availability of the adequate resource in Stack Overflow. If the community support is still evolving in Stack Overflow, then developers can decide to look into other resources. Sometimes community project developed and curated by developers can be an alternative for traditional resources. For example, Rust was a community project of concerned developers. After strong positive feedback, it was donated and has been part of official Rust documentation from Rust 1.25.

\indent \textbf{Language owners:} Our study identifies the significant difference in answer interval between two phases of new languages. As support for the developers in the starting stages is likely to play a significant role in the overall acceptance of that language, owners should provide extensive support during that time. Another option for new languages that are currently in the design stage can be to use the community base of some matured language by carefully selecting predecessor language. Moreover, new languages can try to fill the gap in supporting materials by developer-friendly documentation with detail example. We observed that the issue and release version influences developers' activity pattern(Table~\ref{table:issue_question relationship}, Table~\ref{table:github_question relationship}, Figure~\ref{fig:Release and user behavior}). Though it is not possible to release a bug-free version, extra care must be taken for a bug-free release and solution of issues in GitHub. A good portion of questions in Stack Overflow seeks for clarification of the documentation. Owners should take extra care to prepare documentation suitable for developers of all levels. 

\indent \textbf{Stack Overflow:} Small community size can disrupt the growth of a language. In our study, we have found that the new languages have a small number of experts or active developers in Stack Overflow. To support the growth of a language which has a few expert developers,  Stack Overflow should refine their strategy. According to the current policy, the stack overflow focuses on the expert developers. However, to support new languages, they should encourage developers from all levels to take part in answering questions. It supports the findings of Srba et al.\citep{Srba2016} where they suggested Stack Overflow replace the current question-oriented policy with answer-oriented policy.

\indent \textbf{Researchers:} The topics like parallel execution, mutability, memory allocation are common in the difficult topics of new languages. It proves the limited availability of knowledge in these topics. Researchers may pay attention to these topics as these are an important part of a developer's work practices.
