\section{Related Works}
\label{sec:Related Works}
There have been many works on Stack Overflow data, analyzing trends and characteristics. Barua et al.\citep{Barua2012} have investigated the question ``What developers are asking?" in their work.
Rosen and Shihab\citep{Rosen2015} had a similar work focusing on mobile developers while Bajaj et al.\citep{Bajaj2014}  focused on web developers.

Hart and Sharma\citep{Hart2014} had suggested considering user reputation, the social reputation of answerer and post length to judge post quality.

Reboucas et al.\citep{Reboucas2016} have compared the data from Stack Overflow with opinions of 12 Swift developers to answer three research questions - common problems faced by Swift developers, problems faced by developers in the usage of `optionals,' and error handling in Swift. They used Latent Dirichlet Allocation (LDA) to identify the topics from questions of Stack Overflow and then cross-checked the findings by interviewing Swift developers. These are different from our research questions.

Zagalsky et al.\citep{Zagalsky2016} have analyzed the R language using data from both Stack Overflow and R-help mailing list. They focused on the participation pattern of users in the two communities. They collected user information who are active in both sites and later mining on their activities (questions, answers) they tried to answer how communities create, share and curates knowledge. Vasilescu et al.\citep{Vasilescu2014} have compared popularity and user activity level between StackOverflow and R-help mailing list. They followed a similar approach like Zagalsky et al.\citep{Zagalsky2016} by identifying active users in both communities. They have some interesting finding on the decreasing popularity of mailing list and influence of reputation system in Stack Overflow. Their work is mainly focused on identifying user behavior of these communities.

Tausczik et al.\citep{TausczikWC17} measured the effect of crowd size on Stack-Exchange question quality. They have found that among question audience size, contributor audience size and topic audience size, contributor audience size has a higher effect on solution quality. They have classified the problems into three problems: error problems, how to problems and conceptual problems. Error problems are very specific, as a result, no matter how much the audience size is 25\% problems are never solved. Large audience provides a diverse solution which is critical for how to problems. Conceptual problems are trickier and rarely solved with a small audience.

Srba et al.\citep{Srba2016} had discussed the reason behind the increasing failure and churn rate of Stack Overflow. In their work, criticizing the existing automatic deletion and classification of posts, they introduced a new reputation system. They also suggested to follow answer oriented approach instead of the current asker-oriented approach. Instead of focusing on the highly expert users, Stack Overflow should engage users of all levels.