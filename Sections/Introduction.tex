\section{Introduction}
\label{sec:introduction}
After the release of a new programming language, it takes time for the developers to get acquainted with that language. The developers who work on the new languages are likely to face problems that are similar to the solved problems of mature languages. Detailed knowledge of the growth of resources of a new language can help prevent these issues from appearing again for another language. Moreover, earlier releases of new languages often contain bugs. Developers of the new languages often feel the absence of a library or feature that has already been available in other languages. It is also possible that low-quality documentation and absence of resources in question-answer (QA) sites make the skill growth of developers of new languages slower.

\iffalse
To better help developers of new languages, it is imperative to know how the resources of a language build up in QA site, the impact of language bug in software development and the effects of the absence of a feature. The quality of the support provided in the early stages plays a pivotal role in the acceptance of the language. Despite that, we often notice the absence of quality support for new languages in starting years. To fill the gap in support, language owners can offer extensive support period for new languages, but this is not the only factor contributing to the growth of languages. Identification of all the factors contributing to the growth of language may help to build a collaborative model to support the growth of new languages. In addition to helping the developers, this type of study will help language designers in selecting features, designing documentation and also determining extending support period for developers.
\fi 

To make software development easy, maintainable, robust, and performance-guaranteed, new programming languages are being introduced. For example, Swift was introduced in June 2014 as an alternative to Objective-C to achieve better performance in all the aspects mentioned above. At the initial stage of its lifetime, a programming language is likely to have constraints of resources, and consequently, developers using these languages face additional challenges. Naturally, the developers seek help from community experts of question-answering (QA) sites such as Stack Overflow (SO). Hence, it is expected that the discussions on issues related to a new language in SO represents the growth of that language and also exposes the demands of the development community who use that language. 

To the best of our knowledge, there is yet to be any Software Engineering research that focuses on the specific characteristics of the new languages by mining relevant discussions from SO. In this study, we would like to fill this gap analyzing the discussions on Swift, Go, and Rust that were the most popular languages introduced after the inception of SO (2008). Hence, the evolution right from the beginning of these languages is expected to be reflected in SO. From now on, by the \emph{new language},  we imply Swift, Go and Rust languages. We also match the SO discussions with the relevant activities in GitHub whenever appropriate. 

The primary objective of this research is to study different characteristics of the discussions in SO relevant to these new languages and observe their journey towards maturity. On this goal, our specific research questions are as follows.
\begin{itemize}
\item \textbf{RQ1.} What are the difficult topics in the questions of new languages in Stack Overflow?
\item \textbf{RQ2.} When can we expect the availability of adequate resource of the new languages in Stack Overflow?
\item \textbf{RQ3.} Is there any relation between the growth of a language and developers' activity pattern of that language?
\item \textbf{RQ4.} What are the characteristics of answer pattern for new languages in Stack Overflow?
\item \textbf{RQ5.} Are the questions of new languages in Stack Overflow answered mostly by the developers of predecessor languages?
\end{itemize}

The motivation behind investigating the first research questions is to help the owner/sponsor of these languages to design better features and documentation which would eventually benefit the developers. The general software developers or students can receive insight into how to prepare themselves to work on these languages. Moreover, the last three research questions would cater to the academic interest of researchers by presenting interesting parameters of evolution pattern of new languages and their reflection in SO.


The major findings of the study are: (i) (i) the difficult topics of new languages are quite common, (ii) the time when adequate resources are expected to be available vary from language to language, (iii) the unanswered question ratio increases regardless of the age of the language, (iv) a new language is benefited from its predecessor language and (v) there is a relationship between developers' activity pattern and growth of the language.

